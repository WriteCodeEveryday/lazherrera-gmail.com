\documentclass{article}
\usepackage[utf8]{inputenc}

\usepackage{graphicx}
\usepackage{longtable}
\graphicspath{ {./} }


\title{Private Extractor Export}
\begin{figure}[t]
\includegraphics[width=0.1\textwidth]{ic_launcher.png}
\centering
\end{figure}
\author{Infected Contact Report} 
\date{$date}

\begin{document}

\maketitle


\noindent\fbox{%
    \parbox{\textwidth}{%
       \textbf{What does risk mean?}
       In this context, risk is defined as follows. \\
       You are at risk if you visited a location where an infected person was.
       \centering
       
       \begin{itemize}
        \item \textbf{Green}: You were not at any possible known location within 24 hours.
        \item \textbf{Yellow}: You were at a possible location within 24 hours.
        \item \textbf{Red}: You were at a possible location within 3 hours.
        \end{itemize}
        \centering
		
		\textbf{Do not use this application to make medical decisions.}\\
		\textbf{This application is not a replacement for a medical professional.}\\
    }%
} \\ [5ex]


\noindent\fbox{%
    \parbox{\textwidth}{%
        \textbf{How do I decrease my risk?}
        Follow the WHO / CDC guidelines.
        \centering
        
        \begin{itemize}
        \item \textbf{Wash your hands frequently (for at least 20 seconds).}
        \item \textbf{Maintain social distancing (one meter or three feet).}
        \item \textbf{Cover your mouth and nose when you cough or sneeze.}
        \item \textbf{Avoid touching eyes, nose and mouth.}
        \end{itemize}
        \centering
		
		We're not doctors here, we're nerds, listen to them!
    }%
} \\ [5ex]

Private Extractor is built on the hard work of the open source community and using the following tools: \textbf{Android Debug Bridge, Android Backup Extractor, JLR, Gson, GeoTools, tar, MIT PrivateKit, and Miktex Portable Edition}. Data sourced from \textbf{nCoV-2019 Data Working Group}. Created by Lazaro Herrera.

\pagebreak

\small
\begin{longtable}[c]{| c | c | c | c |}
    \hline
    \multicolumn{4}{| c |}{GPS Coordinates Analyzed}\\ [0.5ex]
    \hline
    Latitude & Longitude & Date / Time & Risk \\ [0.5ex]
    \endfirsthead
    
	\hline
    \endhead
    
    \hline
     \endfoot
    
     \hline
     \endlastfoot
    
     \hline\hline
     #foreach( $point in $points )
        {$point[0]} & {$point[1]} & {$point[2]} & {$point[3]} \\ 
    	\hline
     #end 
\end{longtable}

\pagebreak

\begin{itemize}
   \item  Export Log
   \begin{itemize}
	#foreach( $entry in $log )
		\item	{$entry}
	#end 
	\end{itemize}
\end{itemize}

\end{document}
